\documentclass[10pt,]{beamer}
\usepackage{beamerthemeblackboard}
\usepackage{graphics}
\usepackage{emerald}
\usepackage[T1]{fontenc}

\title{Uso de tecnologias em sala de aula: ferramentas, processos e motivações}
\author{Melissa Weber Mendonça} 
\institute{\small{Universidade Federal de Santa Catarina}}
\date{{1\textordfeminine\ Feira de Matemática da UFSC \textemdash\ 2019}}


\begin{document}

%set the font to Augie (emerald font).
%Comment out this line to only use Augie
%for titles, etc
%\ECFAugie

%title frame
\begin{frame}[plain]
\maketitle  
\end{frame}

% \begin{frame}{Resumo}
% Apesar da utilização constante dos telefones celulares, internet e redes sociais, nas escolas ainda é incipiente a utilização de tecnologias como ferramentas de ensino-aprendizagem, mesmo nas disciplinas como a matemática, intimamente ligada com o desenvolvimento da ciência da computação, linguagens de programação e resolução computacional de problemas em diversas áreas da ciência e das engenharias.

% Neste trabalho, discutimos algumas ideias para o papel da tecnologia em sala de aula, levantando questões sobre que ferramentas usar, em que contextos e qual é a importância que os currículos de educação superior em licenciatura e bacharelado dão para o desenvolvimento desses temas.
% \end{frame}

\begin{frame}
  \begin{center}
    Qual é o objetivo que temos ao ensinar matemática aos nossos alunos?
    \vfill
    \only<2->{Nossa maneira de ensinar atinge esse objetivo? Em que mundo nossos estudantes vivem/viverão?}
  \end{center}
\end{frame}

\begin{frame}{}
  \begin{center}
    Você usa a tecnologia em sala de aula? Como?
  \end{center}
\end{frame}

\begin{frame}{}
  \begin{center}
    Qual o tempo que podemos ``gastar'' para aprender uma ferramenta nova de trabalho?
  \end{center}
\end{frame}

\begin{frame}{Seymour Papert}
  \framesubtitle{(MIT Media Lab)}
  \begin{columns}
    \column{6cm}
    ``(...) as crianças não são contêineres onde deve ser depositado o conhecimento, mas construtoras ativas de conhecimento, pequenos cientistas que estão sempre testando suas teorias sobre o mundo''
    \vfill
    \textemdash\ \href{https://super.abril.com.br/tecnologia/a-maior-vantagem-competitiva-e-a-habilidade-de-aprender/}{Fonte: Entrevista (Super Interessante)}
    \column{5cm}
    \includegraphics[width=4cm]{Seymour_Papert.png}
  \end{columns}
\end{frame}

\begin{frame}{Mitchel Resnick}
  \framesubtitle{(MIT Media Lab)}
  \begin{columns}
    \column{5cm}
    \includegraphics[width=5cm]{Mitchel_Resnick.jpg}
    \column{6cm}
    ``(...) em nosso trabalho de pesquisa no MIT, pensamos que a tecnologia deve levar o aluno a ser um pensador criativo, se desenvolvendo por meio de trabalhos coletivos que envolvam a experimentação de novas formas de se relacionar com o mundo.''
    \vfill
    \textemdash\ \href{https://novaescola.org.br/conteudo/905/mitchel-resnick-a-tecnologia-deve-levar-o-aluno-a-ser-um-pensador-criativo}{Fonte: Entrevista (Nova Escola)}
  \end{columns}
\end{frame}

\begin{frame}{Como?}
  ``É necessário que o professor de matemática organize um trabalho estruturado através de atividades que propiciem o desenvolvimento de \textbf{exploração informal e investigação reflexiva} e que não privem os alunos nas suas iniciativas e controle da situação. O professor deve projetar desafios que estimulem o questionamento, a colocação de problemas e a busca de solução. Os alunos não se tornam ativos aprendizes por acaso, mas por desafios projetados e estruturados, que visem a exploração e investigação.''
  \vfill
  Richards, em \cite{richards}
\end{frame}

% \begin{frame}{Como?}
%   Graficar uma função em um software não necessariamente tem o papel de verificar ou corrigir um gráfico feito inicialmente no papel. Porém, pode servir como pretexto para uma discussão de interpretações e para o questionamento de decisões tomadas ao longo da resolução de uma questão sem o uso do computador. O objetivo deve ser ``usar o computador para promover aprendizagem matemática sólida o suficiente para permanecer e se transferir para outras situações \textemdash\ \emph{mesmo sem o apoio da máquina}.'' \cite[p.65]{profmat2013}
% \end{frame}

% \begin{frame}{Como?}
%   ``(...) temos como meta uma incorporação efetiva à prática docente \textemdash\ sem que o computador se reduza a um mero adereço, alegórico para a abordagem, e que a aula no laboratório de informática adquira um caráter de curiosidade, desconectada da aula ``de verdade'', aquela com quadro negro e giz. (...) a questão a considerar não deve ser como recursos computacionais podem ser anexados a abordagens previamente estabelecidas, e sim como sua \emph{integração à prática docente pode viabilizar a produção de novas abordagens, possibilitando reestruturação da ordem e das conexões entre os conteúdos, e criando novas formas de explorar e de aprender Matemática}.'' \cite[p.VII-VIII]{profmat2013}
% \end{frame}

% \begin{frame}{Como?}
%   ``É fundamental que os alunos construam um senso crítico para o resultado do computador, sustentado pela consciência de que a máquina não é isenta de limitações, e de que seus resultados devem ser entendidos à luz de argumentos matemáticos (e não ao contrário).'' \cite[p.X]{profmat2013}
% \end{frame}

\begin{frame}{Desafios}
  \begin{itemize}
  \item Barreiras de ordem prática: carência de recursos materiais; resistências políticas por parte das direções escolares; falta de tempo. ``Entretanto, (...) \emph{justamente para se munir de contra-argumentos para tais obstáculos, é importante conhecer essas possibilidades.}'' \cite[p.VII]{profmat2013}
  \item Novas formas de avaliação
  \item Insegurança dos professores em relação às ferramentas; medo de cometer erros
  \item ``A informática por si só não garante esta mudança, e muitas vezes se pode ser enganado pelo visual atrativo dos recursos tecnológicos que são oferecidos, mas os quais simplesmente reforçam as mesmas características do modelo de escola que privilegia a transmissão do conhecimento.'' \cite{gravina}
  \end{itemize}
\end{frame}

\begin{frame}{Possibilidades}
  \begin{itemize}
  \item Construtivismo aplicado! (Piaget+Papert)
  \item Apresentação de assuntos de maneira dinâmica, permitindo aos alunos \emph{manipular} os objetos matemáticos
  \item Acesso a informações e organização de conteúdos em hipertexto
  \item Aprendizagem assíncrona: Ampliação da sala de aula e possibilidade de revisão para o aluno no caso de materiais disponíveis online (video-aulas, textos)
  \item Devemos separar a aprendizagem da linguagem e a apresentação de assuntos em sala de aula; o professor deve saber programar, o aluno não precisa.
  \item Devemos explorar as limitações da máquina como oportunidade de aprendizagem
  \end{itemize}
\end{frame}
  
% \item Os graficadores podem ser ferramentas para a articulações múltiplas entre diferentes formas de representação de funções, por exemplo, entre a Tabela, a Fórmula e o Gráfico\cite[p.59]{profmat2013}.

% \item Escalas logarítmicas, coordenadas polares

% \item O gráfico de uma função é o ``lugar geométrico dos pontos do plano cartesiano cujas coordenadas verificam a lei de formação da função. Em geral, os alunos aprendem tantos procedimentos para traçar gráficos em casos particulares, que essa noção fundamental fica em segundo plano.''\cite[p.171]{profmat2013}.

\begin{frame}{Ideias de Atividades}
  \begin{itemize}
    \item A inclusão digital é uma realidade?
    \item Devemos priorizar atividades coletivas
    \item Atividades devem apontar para reflexões sobre os conceitos matemáticos, independente das tecnologias empregadas
    \item Os recursos computacionais mais simples, como as calculadoras, permitem ao professor a utilização de dados numéricos realistas (e não só inteiros ou ``bonitos'') 
  \end{itemize}
\end{frame}

\begin{frame}{Ideias de Atividades: Ensino Fundamental}
  \begin{itemize}
  \item Manipulação de operações aritméticas com calculadora
  \item Utilização de câmera e aplicativos auxiliares para medir e manipular objetos reais em geometria (calcular distâncias, alturas, projeções, proporções)
  \item Utilização de relógio, fuso horário, temporizador para a manipulação e conversão de unidades de tempo
  \item Jogos educativos e lógicos
  \end{itemize}
\end{frame}

\begin{frame}{Ideias de Atividades: Ensino Médio}
  \begin{itemize}
  \item Interpretações de dados do cotidiano: aproximações, interpolações, estatísticas, operações, gráficos etc
  \item Testar eixos, escalas e visualizações diferentes pode ser interessante para explorar o cuidado que se deve ter ao observar e tentar interpretar e generalizar dados incompletos advindos de gráficos
  \item Linguagens de programação, planilhas eletrônicas: formalismo da simbologia algébrica (sintaxe e semântica)
  \item Funções em ambientes de geometria dinâmica podem ser exploradas como famílias a partir de parâmetros
  \item Otimização
  \end{itemize}
\end{frame}

\begin{frame}{Alguns recursos}
  \begin{itemize}
  \item Calculadora!
  \item Mathpix/Photomath
  \item Geogebra
  \item Maxima
  \item Linguagens de programação geral: Python
  \item Lego
  \end{itemize}
\end{frame}

\section{Referências}
\begin{frame}{Referências}
  \bibliographystyle{plain}
  \bibliography{refs_fermat}
  % PIAGET, Jean. Aprendizagem e Conhecimento, em Piaget, P. & Gréco, P.,
  % Aprendizagem e Conhecimento, Freitas Bastos, Rio de Janeiro, 1974.
  % Léa da Cruz Fagundes in Nova Escola On-line , edição número 172, maio de 2004,
  % “Podemos vencer a exclusão digital”.
\end{frame}

\end{document}
